\documentclass[a5paper,pagesize,twoside,10pt,headings=small]{scrreprt}

\usepackage[utf8]{inputenc}
\usepackage[ngerman]{babel}
\usepackage[T1]{fontenc}
\usepackage{guitar}
\usepackage{multicol}
\usepackage[top=1.2cm, bottom=1.5cm, left=2.0cm, right=0.5cm]{geometry}
\usepackage{lscape}

\pagestyle{plain}

\makeatletter
\newcommand\ignore[1]{}
\setlength{\parskip}{-3ex}

% \songheader{Titel}{Autor}
\newcommand\songheader[3][]{%TODO: Seitenumbrüche sind hier genau falsch.
  \clearpage%
    % \par{\Large\bf\sffamily #2}\\[0.2\baselineskip]%
    \subsection*{\centering #2}\vspace{0.5\baselineskip}%
    \addcontentsline{toc}{subsection}{#2}
    \centering \textit{#3}\\%
    \vspace{\baselineskip}\par%
}

%\newcommand\songfooter[1]{\par\begin{flushright}\tiny #1\end{flushright}}
\newcommand\songfooter[1]{\vfill\par\begin{flushright}\tiny #1\end{flushright}}
%% Abstand vergrößern zwischen Strophen, der ist sonst kaum vom Zeilenabstand zu unterscheiden.
\renewcommand\guitarEndPar{\par\vspace{1.5\baselineskip}}
\newcommand\Refrain{\textbf{Refrain}\vspace{-1.0\baselineskip}}
\newcommand\Refraindef{\textbf{Refrain:}}
\newcommand\leftrepeat{{\large $|\hspace{-0.2ex}|\hspace{-0.8ex}:\,$}}%TODO
\newcommand\rightrepeat{{\large $\,:\hspace{-0.8ex}|\hspace{-0.2ex}|$}}%TODO
\newcommand\Repeat[1]{\leftrepeat #1 \rightrepeat}
\makeatother

\newcommand\todo{TODO}
\newcommand\done{DONE}
\setlength{\footskip}{2.0\baselineskip}

\begin{document}
\tableofcontents
\chapter*{Rock/Pop}

\section*{Deutsch}
%TODO: Mini-Inhaltsverzeichnis???
\songheader{Als ich fortging}{Karat}
\begin{guitar}
[Bm]Als ich fortging war die [Em]Straße steil, [A]kehr' wieder [D]um.
[G]Nimm an ihrem Kummer [A]teil, mach sie [Bm]heil.

[Bm]Als ich fortging war der [Em]Asphalt heiß, [A]kehr wieder [D]um.
[G]Red' ihr aus um jeden [A]Preis was sie [Bm]weiß.

[G]Nichts ist [D]unendlich, so [Em]sieh das doch [Bm]ein.
[G]Ich weiß du willst unendlich [A]sein, schwach und [Bm]klein.
[G]Feuer brennt [D]nieder, wenn's [Em]keiner mehr [Bm]nährt.
[G]Kenn' ja selber was dir [A]heut wider[Bm]fährt.

[Bm]Als ich fortging war'n die [Em]Arme leer, [A]kehr' wieder [D]um.
[G]mach's ihr leichter einmal [A]mehr, nicht so [Bm]schwer.

[Bm]Als ich fortging kam ein [Em]Wind so schwach, [A]warf mich nicht [D]um.
[G]Unter ihrem Tränen[A]dach war ich [Bm]schwach.

[G]Nichts ist [D]unendlich, so [Em]sieh das doch [Bm]ein.
[G]Ich weiß du willst unendlich sein, [A]schwach und [Bm]klein.
[G]Nichts ist von [D]Dauer, was [Em]keiner recht [Bm]will.
[G]Auch die Trauer wird da sein, [A]schwach und [Bm]klein.
\songfooter\done
\end{guitar}


\songheader{Marzipan}{(Norbert und die Feiglinge)}
\begin{guitar}
{\tiny(Melodie: Lady in Black von Uriah Heap)}
[Am]Ich ess so gerne Marzipan
[Am]Es geht nichts über Marzipan
[G]Ich denk den ganzen Tag daran,
[Am]Nur dafür schlägt mein Herz.
[Am]Ich aß grad einen Pelikan aus allerfeinstem Marzipan
Da [G]plötzlich durchfuhr meinen Körper ein [Am]stechender Schmerz.

[Am]Aa[G]aa[Am]aa[Am]aa[G]h...[Am].

Verdammt, da war ein Loch im Zahn
Das kommt von zuviel Marzipan.
Ich rief sofort den Doktor an uns sprach von meiner Qual.
Der Doktor war ein weiser Mann
Man sah's ihm an der Brille an
Er sagte: "`Na dann öffnen Sie den Mund und sagen Sie mal"'

Aaaaaah....

Er brummelte:"`Mein lieber Schwan"'
Und irgendwas vom Schlendrian
Er fing sofort zu bohren an mit furchtbarem Radau
Es tat jedoch fast gar nicht weh
Ich dachte"`Kerl, was biste zäh
Und tapfer". Doch dann traf er einen Nerv und zwar genau.

Aaaaaah....

Ach bitte Doktor sei human
Ich hab doch keinem was getan
Ich sage auch dem Marzipan für alle Zeit ade
Er füllte ihn mit Amalgam, den Rest von meinem Backenzahn
dann fragt' er mich:"`Na junger Mann, wo tut es denn jetzt noch weh?"'

-"`Daaaaaa..."'
-"`Hier?"'
-"`Jaaaaaah..."'

"`Den müssen wir wohl ziehen den Zahn.
Haben wir noch ein' Termin im Plan?
Brigitte, stimmen sie das mal bitte
Mit dem Herrn hier ab"' -"`Jaha!"'
Das war die Zahnarzthelferin
Die war so schön, ich war ganz hin
Und weg, sie fragte zärtlich, welche Blutgruppe ich hab.

-"`Aaaaaaa...."'
-"`Und die Krankenkasse?"'
-"`Aaaaaa... AOK!"'

Das war ja fast wie im Roman
Verliebt bin ich nach Haus gefahrn
Sie hatte sich für meine Krankenkasse interessiert
Seither verlier ich Zahn um Zahn,
Ich eß noch viel mehr Marzipan
Da jedes Zahnweh mir ein Wiedersehen gerantiert
Und wenn ich jetzt hier sitz und schrei
Dann ist das völlig einerlei
ich leide gern, mein Leid ist durch die Liebe motiviert.

Aaaaaaaaah...
\end{guitar}

\songheader{Ich denk, es war ein gutes Jahr}{Reinhard Mey}%%TODO
\begin{guitar}
Der Raureif legt sich vor mein Fenster,
Kandiert die letzten Blätter weiß.
Der Wind von Norden jagt Gespenster 
aus Nebelschwaden über's Eis,
Die in den Büschen hängenbleiben 
an Zweigen wie Kristall so klar.
Ich hauche Blumen auf die Scheiben 
und denk', es war ein gutes Jahr!
Ich hauche Blumen auf die Scheiben 
und denk', es war ein gutes Jahr!

Sind ein paar Hoffnungen zerronnen? 
War dies und jenes Lug und Trug?
Hab' nichts verloren, nichts gewonnen, 
so macht mich auch kein Schaden klug.
So bleib ich Narr unter den Toren, 
hab' ein paar Illusionen mehr,
Hab' nichts gewonnen und nichts verloren, 
und meine Taschen bleiben leer.
Hab' nichts gewonnen und nichts verloren, 
und meine Taschen bleiben leer.

Nichts bleibt von Bildern, die zerrinnen. 
Nur eines seh' ich noch vor mir,
Als läg' ein Schnee auf meinen Sinnen, 
mit tiefen Fußstapfen von dir!
Mir bleibt noch im Kamin ein Feuer 
und ein paar Flaschen junger Wein.
Mehr Reichtum wär mir nicht geheuer 
und brächte Sorgen obendrein.
Mehr Reichtum wär mir nicht geheuer 
und brächte Sorgen obendrein.

Du kommst den Arm um mich zu legen,
Streichst mit den Fingern durch mein Haar:
\Repeat{"`Denk dran ein Holzscheit nachzulegen ... Ich glaub', es war ein gutes Jahr!"'}
ladadada.....
\end{guitar}




\songheader{Du hast mir gesagt}{Element of Crime}
\begin{guitar}
Ich [A]liege träge in der [D]Sonne, und [A]sehe, wie der Fluß [D]vorüberfließt
[A]Du hast mir gesagt, daß man [D]nichts erzwingen soll
Und ich [A]warte hier, und ich [D]warte hier
Auf das Glück und den gestrigen [A]Tag[D].

Ich freunde mich an mit einer Spinne und schaue ihr bei der Arbeit zu
Du hast mir gesagt, daß man Tiere lieben soll
Und ich warte hier, und ich warte hier
Auf das Glück und den gestrigen Tag

[D]Wenn du mich liebst, wirst du mich [A]finden
[D]Verliebte gehn hier immer an den [A]Fluß
Und [D]du hast mir gesagt, das alles [A]kommt, wie es kommen muß
Und ich [D]warte hier, und ich [D]warte hier, auf das [D]Glück und den gestrigen [A]Tag

Ich werfe Steine in die Strömung, und schaue ihnen beim Versinken hzu
Du hast mir gesagt, daß man sich nützlich machen soll
Und ich warte hier, und ich warte hier, auf das Glück und den gestrigen Tag

Wenn du mich liebst ...

Ich würde gerne endlich mal was essen
und die Augen fallen mir von selber zu
Du hast mir gesagt, daß man nicht locker lassen soll
Und ich warte hier, und ich warte hier, und ich warte hier,
auf das Glück und den gestrigen Tag

Wenn du mich liebst ...
\end{guitar}

\songheader{Wer ich wirklich bin}{Element of Crime}
\begin{guitar}
[Em]In meinem Schädel wohnt ein Tier, das trampelt alles kurz [Am]und klein
[Em]Was ich mir vorgenommen hab für einen lieben [Am]langen Tag
[D]Wozu bewegen, [H]ich weiß ja nicht [Em]wohin
[D]Manchmal wüsst ich gern, [H]wer ich wirklich [Em]bin

In meiner Kehle wohnt ein Tier, das frisst die klugen Worte auf
Die ich mir beigebogen hab für einen lieben langen Tag
Was soll ich sagen, wenn da keine Worte sind
Manchmal wüsst ich gern, wer ich wirklich bin

In meinem Herzen wohnt ein Tier, das beißt die edle Regung tot
Die ich mir abgerungen hab für einen lieben langen Tag
Wozu Gedanken, wenn die doch nur böse sind
Manchmal wüsst ich gern, wer ich wirklich bin

In meinen Augen wohnt ein Tier, das sieht mich immer hungrig an
Als ob's mich aufgehoben hat für einen lieben langen Tag
Was soll werden, wenn es die Oberhand gewinnt
Manchmal wüsst ich gern, wer ich wirklich bin
\end{guitar}

\songheader{Delmenhorst}{Element of Crime}
\begin{guitar}
[G/C (4x)] 

[G]Ich bin jetzt immer da, wo [C]du nicht bist / [G]Und das ist immer Delmen[C]horst
[Am]Es ist schön, wenn's nicht [G]mehr weh tut / [Am]Und wo zu sein, wo du nie [D]warst
[Hm]Hinter Huchting ist ein [C]Graben / [D]Der ist weder breit noch [Em]tief
[G]Und dann kommt gleich Getränke-[Am]Hoffmann,
[C]Sag Bescheid, wenn du mich [G]liebst

Ich hab jetzt Sachen an, die du nicht magst
Und die sind immer grün und blau
Ob ich wirklich Sport betreibe / Interessiert hier keine Sau
Hinter Huchting ist ein Graben / Der in die Ochtum sich ergießt
Und dann kommt gleich Getränke Hoffmann
Sag Bescheid, wenn du mich liebst

Ich mach jetzt endlich alles öffentlich / Und erzähle, was ich weiß
Auf der Strasse der Verdammten / Die hier Bremer Straße heißt
Hinter Huchting ist ein Graben / In den sich einer übergibt
Und dann kommt gleich Getränke Hoffmann
Sag Bescheid, wenn du mich liebst

Ich bin jetzt da, wo ich mich haben will / Und das ist immer Delmenhorst
Erst wenn alles scheißegal ist / Macht das Leben wieder Spaß
Hinter Huchting ist ein Graben / Der ist weder breit noch tief
Und dann kommt gleich Getränke Hoffmann
\Repeat{[Cm]Sag Bescheid, [G]wenn du mich liebst}$_3$
\end{guitar}

\songheader{Straßenbahn des Todes}{Element of Crime}
\begin{guitar}
[Am]Wo die alten Leute [Em]immer / [Am]Schimpfend an der Ecke [Em]stehen
[Am]Werd ich mich dazu [Em]gesellen / [Am]Und Zigaretten für sie [H7]drehen

[Am]Und grad wenn [D]wir uns richtig ei[G]nig sind,
Dass [C]alles immer [Am]schlimmer wird,
Wirst [D]du vorüber [G]gehen [G7] 
[Am]Wo die Neu[D]rosen wuchern
[G]Will ich Landschafts[C]gärtner sein
Und [H7]dich will ich endlich wieder[Em]sehen

Vor dem Edeka des Grauens / Wo wer freundlich ist verliert
Halt ich jeden Abend Wache / Auf dem Container für Papier

Und wenn du kommst, dann sehe ich 
dich von weitem schon
Und kann schnell zu den Einkaufswagen gehen
Wo die Neurosen wuchern
Will ich Landschaftsgärtner sein
Und dich will ich endlich wiedersehen

In die Straßenbahn des Todes / Die heulend sich zum Stadtrand quält
Werd ich mich klaustrophobisch zwängen / Weil auch die kleine Geste zählt

Und wenn du einsteigst
Bin ich lang schon eng befreundet
Mit den Leuten die auf meinen Füßen stehen
Wo die Neurosen wuchern
Will ich Landschaftsgärtner sein
\Repeat{Und dich will ich endlich wiedersehen}
\end{guitar}

\songheader{Bring den Vorschlaghammer mit}{Element of Crime}
\begin{guitar}
[Am]Siehst du diesen Teller, den hab ich aus [C]Florenz
[Am]und der alte Benz läuft immer noch wie [C]neu
na und [F]wie ich mich da freu
und der [Em]Plattenspieler, der wird heute [Dm]nicht mehr [G]gebaut
der war mir [C]lange Jahre [A]treu.

Bring den [Dm]Vorschlaghammer [G]mit, wenn du [C]heute abend [Am]kommst
dann [Dm]hauen wir [G]alles kurz und [C]klein[A],
der ganze [Dm]alte Schrott muß raus, [G]und neuer [C]Schrott muß rein[Am],
bis [Dm]morgen muß der ganze Rotz [G]verschwunden [C]sein!


Der Aschenbecher da, den hab ich mal geklaut
ich glaub, das war in einem griechischen Lokal
und das Plattenregal 
habe ich selbstgebaut, das war normal, der Herd war gekauft
und die anderen Möbel auch

Bring den Vorschlaghammer mit, wenn du heute abend kommst
dann hauen wir alles kurz und klein
der ganze alte Schrott muß raus und neuer Schrott muß rein
bis morgen muß der ganze Rotz verschwunden sein

Der Computer ist auf dem allerneuesten Stand
da ist noch Pfand auf den Flaschen, die in der Küche stehen, 
da will ich bald mal Scherben sehen
und der Bücherwand, für die ein halber Wald einmal starb,
schlägt die letzte Stunde bald
\end{guitar}


\songheader{Ich war nicht dabei}{Element of Crime}
\begin{guitar}
[Em]Erst war es Silvester und später neues Jahr
Und sie [Am]schrien mir ins Ohr, wie es später wirklich war
Da sprangen [H]alle auf die Tische, und ich war nicht [Em]dabei.
Die [Em]Leiber so verschwitzt und die Augen so verstrahlt
Und [Am]kurz nachdem ich ging hat keiner mehr bezahlt
Die tranken [H]alles, was nicht weglief, und ich war nicht da[Em]bei.
[F]Wo ich da wohl grade [G]war -- das wissen [B]eigentlich nur wir [H]zwei!

Zuerst waren's über Hundert, und später nur noch vier
Doch die marodierten weiter auf ein allerletztes Bier
Mein Name war ihr Schlachtruf, und ich war nicht dabei.
Später gab es Ärger und ordentlich Polizei
Bloß hat das keiner mehr gemerkt, die waren alle viel zu high
Das war dann irgendwie auch schade, und ich war nicht dabei.
Wo ich da wohl grade war -- das wissen eigentlich nur wir zwei!

Erst war es noch dunkel, und später wieder hell
Und später wieder dunkel, und dann war's wieder hell
Das war dann auch nicht mehr so wichtig, und ich war nicht dabei.
Am Besten war's am Ende, da hat keiner mehr gelacht
Da floß der Reue heiße Träne, das hat richtig Spaß gemacht
Da waren alle richtig alle, und ich war nicht dabei
Wo ich da wohl grade war -- das wissen eigentlich nur wir zwei!
\end{guitar}


\songheader{Mit Dir allein}{Element of Crime}
\begin{guitar}
Mit [Gm]zitternden Händen sitz ich am [Am]Fenster
Und [Gm]rauche als kriegt ich's be[Am]zahlt
Im [Fmaj7]Hinterhof schreit eine [Gm]Katze
Und [B]darüber sieht der [A]Himmel aus wie gemalt
Jetzt bloß keine [Gm]Stille, das könnte [Am]fatal sein
Du kramst nach [Gm]Gläsern im Bücherre[Am]gal
In meinem [Fmaj7]Kopf sieht es aus wie in deinem [Gm]Zimmer
Was ich [B]trinken will, fragst du, das ist mir [A]sowas von egal!

Das erste [Dm]Mal mit dir [Gm7]allein[Gm6],
Das erste [Dm]Mal mit dir [Gm]allein[Gm6].

Wir reden wahllosen Schwachsinn
Ich kneife mich heimlich ins Bein
Im Hinterhof zanken die Nachbarn
Und wie ein Vollidiot stopf ich 
Kartoffelchips in mich rein
\Repeat{Das erste Mal mit dir allein}

Sag doch irgendwas Schlaues
Mir fällt leider nichts ein
Im Hinterhof zerplatzt eine Lampe
Und als stiller Komplize
kriecht die Dämmerung zu uns rein
\Repeat{Das erste Mal mit dir allein}
\end{guitar}

\songheader{Wenn der Morgen graut}{Element of Crime}
\begin{guitar}
[G] \ [Em] \ [G] \ [Em] \ 

[G]Kurz vor der ersten Straßen[Em]bahn
[G]Sind alle Häuser finster und [Am]stumm.
[Am]Dreh dich noch einmal nach mir [G]um,
[G]Einmal für [Em]dich, [Em]einmal für [G]mich.

Kurz vor der ersten Straßenbahn
Sind alle Wege öde und leer.
Lauf noch ein bischen neben mir her,
Einmal für dich, einmal für mich.

[C]Wo ist der Gott, der uns [Am]liebt, Ist der [Am]Mensch, der uns [C]traut,
Ist die [C]Flasche, die uns [Am]wärmt, Wenn der [Am]Morgen [G]graut?

Kurz vor der ersten Straßenbahn
Sind die Gedanken müde und schwer.
Ein Stern fällt ins Wasser und der Mond hinterher,
Einmal für dich, einmal für mich.

Wo ist der Gott, ...

[Hm]Dreh dich noch einmal nach mir [Am]um,
[Hm]Lauf noch ein bischen neben mir [Am]her!
Ein [Am]Stern fällt ins Wasser und der Mond hinter[G]her,
[G]Einmal für [Em]dich, [Em]einmal für [G]mich.

Wo ist der Gott, ...
\end{guitar}


\songheader{Und du wartest}{Element of Crime}
\begin{guitar}
Ein [Am]alter Mann steht unten am See
Und be[Dm]wirft die Enten mit [F]Brot vom vorigen [Am]Jahr
Und du [Em]wartest.
Kommt [Am]Zeit, kommt Rat, kommt Vater Staat,
Kommt [Dm]Rente, kommt Ente, kommt [F]Haß auf alles, was [Am]früher war;
Und du [Em]wartest -- auf [Dm]irgendwas
Auf den [E]gestrigen Tag, auf [Am]längeres Haar
Auf den [Dm]Sommer, und darauf, daß [F]einer das Klo [Am]repariert.
[Am]Sogar [G]auf ein Zeichen von [C]ihr.

Ein kleiner Junge steht unten im Hof
Und beweint den Tod eines Schneemanns vom vorigen Jahr
Und du wartest.
Kommt Zeit, kommt Rat, kommt gute Tat,
Kommt gute Fee, kommt Schnee, kommt Haß auf alles, was früher war;
Und du wartest -- auf irgendwas
Auf die Müllabfuhr, einen Platz an der Bar
Auf den Sommer, und darauf, daß einer das Klo repariert.
Sogar auf ein Zeichen von ihr.

Ein alter Zausel betrachtet sich selbst
Und zerschabt sein Gesicht mit Klingen vom vorigen Jahr
Und du wartest.
Kommt Zeit, kommt Rat, kommt Fusselbart,
Kommt Rasierapparat, und vergessen ist alles, was früher war;
Und du wartest -- auf gar nichts mehr
Auf den Sommer nicht, und nicht auf längeres Haar.
Schon gar nicht auf ein Zeichen von ihr.
\end{guitar}


\songheader{Weißes Papier}{Element of Crime}
\begin{guitar}
Ich [Dm]nehm deine Katze und [Dm]schüttel sie aus, bis [Am]alles herausfällt [Am]aus ihr,
was sie [Gm]jemals aus meiner Hand [F]fraß
[B]später klopf ich noch den [A]Teppich aus
Und [Dm]find ich ein Haar von Dir [Dm]darin
so [Am]steck ich es einfach ein[Am].
[Gm]nichts soll dir böse [F]Erinnerung sein
[B]verraten, was ich dir [A]gewesen  bin
Sag [B]nicht, daß das gar nicht nötig [C]wär
den [C7]schmerzhaft wird es erst [F]hinterher
wenn wieder [Am]hochkommt, was früher mal [Dm]war
Dann [Gm]lieber so rein und so [A]dumm sein wie weißes [Dm]Papier

Auch werd ich in Zukunft ein anderer sein, Als der, der ich einmal war
die Hose, die du mir gehäkelt hast
werf ich in den Container der Heilsarmee rein
Ich eß auf dem Fußboden aus der Hand
seh mir jeden Trickfilm im Fernsehn an
alles was du nicht magst, lobe ich mir
Ich werd einfach so rein und so dumm sein wie weißes Papier

Nicht mal das Meer darf ich wiedersehn, wo der Wind deine Haare vermißt
Wo jede Welle ein Seufzer
und jedes Sandkorn ein Blick von dir ist
Am liebsten wär ich ein Astronaut
und flöge auf Sterne, wo gar nichts vertraut
und versaut ist durch eine Berührung von dir
Ich werd nie mehr so rein und so dumm sein wie weißes Papier
\end{guitar}


\songheader{Ein Kompliment/Un Complimento}{Sportfreunde Stiller}
\begin{multicols*}{2}
\begin{guitar}
[D]Chiamati se vuoi,
la fine di un [Am]tragitto lungo,
[C]perfetta sotto ogni [Em]aspetto
[Em]e lieve nei momenti di silenzio,
[D]la cresta dell' [Am]ondata di passione,
in [C]salita la mia spinta in [Em]propulsione.

[D]Ascoltami, [Am]volevo solo dirti che [C]tu sei,
il massimo per [Em]me.
[D]e star sicuro [Am]che hai la stessa idea,
[C]su di me, su di [Em]me.

Chiamati se vuoi,
la mia chill-out-area,
una vacanza che non se ne va,
un frigo pieno di gelati alla spiaggia,
la soluzione se qualcosa va storto,
così preziosa che la tengo nascosta
e così bella
che non vorrei mai farne a meno.

Ascoltami, volevo solo dirti che tu sei,
il massimo per me.
e star sicuro che tu hai la stessa idea
su di me, su di me.

\columnbreak

[D]Wenn man so will,
bist [Am]du das Ziel einer langen Reise. 
Die [C]Perfektion der besten Art und [Em]Weise, 
[Em]in stillen Momenten leise, 
[D]Schaumkrone der [Am]Woge der Begeisterung
[C]bergauf, mein Antrieb und [Em]Schwung.

[D]Ich wollte dir [Am]nur mal eben sagen,
dass du [C]das Größte für mich [E]bist.
[D]und sichergehen, [Am]ob du denn dasselbe
für mich [C]fühlst, für mich [E]fühlst.

Wenn man so will, 
bist du meine Chill-Out Area, 
meine Feiertage in jedem Jahr, 
meine Süßwarenabteilung im Supermarkt. 
Die Lösung, wenn mal was hakt, 
so wertvoll, dass man es sich gerne aufspart, 
und so schön,
dass man nie darauf verzichten mag.

Chorus 2x
\end{guitar}
\end{multicols*}


\newpage\section*{International}
\songheader{Shape of my Heart}{Sting}\vspace{-\baselineskip}
\begin{guitar}
[Fm]He deals the cards as a [Bm]meditation
[Fm]And those he plays never [Bm]suspect
[D/F#]He doesnt play for the [A]money he [C#/F]wins
[D/F#]He doesnt play for the [Fm]respect
[Fm]He deals the cards to find the [Bm]answer
[Fm]The sacred geometry of [Bm]chance
[D/F#]The hidden law of [A]probable [C#/F]outcome
[D/F#]The numbers lead a [Fm]dance

\Refraindef [Fm]I know that the spades are the [Bm]swords of a soldier
[Fm]I know that the clubs are weapons of [Bm]war
[D/F#]I know that diamonds mean [A]money for this [C#/F]art
[D/F#]But thats not the shape of my [Fm]heart

He may play the jack of diamonds, He may lay the queen of spades
He may conceal a king in his hand, While the memory of it fades

\Refrain

And if I told you that I loved you / You'd maybe think there's something wrong
Im not a man of too many faces / The mask I wear is one
Those who speak know nothing / And find out to their cost
Like those who curse their luck in too many places
And those who smile are lost.

\Refrain
\end{guitar}

\songheader{Fields of Gold}{Sting}
\begin{guitar}
[Am]You'll remember me [F]when the west wind moves,
[F]Upon the [F/G?]fields of [C]barley
[Am]You'll forget the sun [F]in his [C]jealous sky
[F]As we walk in the [G]fields of [C]gold

So she took her love / For to gaze awhile / Upon the fields of barley
In his arms she fell as her hair came down / Among the fields of gold

Will you stay with me, will you be my love / Among the fields of barley
We'll forget the sun in his jealous sky / As we lie in the fields of gold

See the west wind move like a lover so / Upon the fields of barley
Feel her body rise when you kiss her mouth / Among the fields of gold

[Am]I never made [F]promises [G]lightly [C],
[Am]And there have been [F]some that I've [G]broken [C],
[Am]But I swear in the [F]days still [G]left [C],
\Repeat{[F]We'll walk in the [G]fields of [C]gold}

Many years have passed since those summer days
Among the fields of barley
See the children run as the sun goes down
Among the fields of gold
You'll remember me when the west wind moves
Upon the fields of barley
You can tell the sun in his jealous sky
\Repeat{When we walked in the fields of gold}$_3$
\end{guitar}




\chapter*{Alte Musik}

\songheader{An hellen Tagen}{Giovanni Gastoldi, "`Balletti a 5 voci"', 1591 \quad deutsch: Peter Cornelius (1824 - 1874)}%TODO
\begin{guitar}
\Repeat{[C]An hellen Tagen, [C]Herz welch ein Schlagen.
[C]Fa la la la la, [G]fa la la [C]la.}
[C]Himmel dann [F]blauet, Auge dann schauet,
[G]Herz wohl den beiden, [F]manches vertrauet.
[C]Fa la la la la la , [G]fa la la [C]la.
 
\Repeat{Beim Dämm'rungsschimmer, Herz du schlägst immer
Fa la la la la, fa la la.}
Ob auch zerronnen Strahlen und Wonnen,
Herz will an beiden still sich noch sonnen.
Fa la la la la la , fa la la la.
 
\Repeat{Wenn Nacht sich neiget Herz nimmer schweiget.
Fa la la la la, fa la la.}
Schlummer mag walten, Traum sich entfalten
Herz hat mit beiden Zwiesprach zu halten.
Fa la la la la la , fa la la la.
\end{guitar}


\songheader[Scribere proposui]{Ad Mortem festinamus}{\todo}%TODO
\begin{guitar}
Scribere proposui de contemptu mundano,
ut degentes seculi non mulcentur in vano.
Iam est hora surgere a sompno mortis pravo.
Ad mortem festinamus, peccare desistamus!

Vita brevis breviter in brevi finietur,
mors venit velociter quae neminem veretur.
Omnia mors perimit et nulli miseretur.
Ad mortem festinamus, peccare desistamus!

Ni conversus fueris et sicut puer factus,
et vitam mutaveris in meliores actus.
Intrare non poteris regnum Dei beatus.
Ad mortem festinamus, peccare desistamus!

Tuba cum sonuerit dies crit extrema,
et iudex advenerit vocabit sempiterna.
Electos in patria, prescitos ad inferna.
Ad mortem festinamus, peccare desistamus!
\end{guitar}

\songheader{Gaudete}{16. Jahrhundert, ``Piæ Cantiones''}%TODO
\begin{guitar}
\Refraindef
[Am]Gaude[G]te, [Am]gaude[C]te! [C]Chris[Em]tus est natus
[G]Ex Ma[Em]ria virgine, [Am]gau[G]de[Am]te!

[Am]Tempus adest gratiæ,
Hoc quod opta[G]bamus,
Carmina lætitiæ
[Em]Devote [Am]reddamus.

\Refrain

Deus homo factus est
Natura mirante,
Mundus renovatus est
A Christo regnante.

\Refrain

Ezechielis porta
Clausa pertransitur,
Unde lux est orta
Salus invenitur.

\Refrain 

Ergo nostra contio
Psallat iam in lustro;
Benedicat Domino:
Salus Regi nostro.

\Refrain
\end{guitar}

\songheader{Gaudete (Potentia Animi)}{Potentia Animi}%TODO
\begin{guitar}
\Refraindef
\Repeat{Gaudete, Gaudete Christus est Natus 
Ex Maria Virginae, Gaudete!}

Cunnilingus Vagina - Fellatio Phallus,
Fantasië Orgie, Multiple Orgasmus!

\Refrain

Sodomië, Pedophil, Exibitionismus,
Necrophile Inzest Sadomasochismus!

\Refrain

Onanië, Petting, Masturbation,
Pornophil, Godmiché, Ejakulation!

\Refrain

Syphilis, Gonorrhoe, Ulcus molle lues
Balanitis Scabies, Peticulus Pubes!

\Refrain
\end{guitar}


\songheader{Gaudeamus igitur}{Christian Wilhelm Kindleben, 1781}
\begin{multicols*}{2}
\begin{guitar}
\Repeat{[G]Gaudeamus [C]igitur
[D]iuvenes dum [G]sumus}
[D]post iucundam [G]iuven[D]tutem,
[D]post molestam [G]senec[D]tutem,
\Repeat{[Em]nos ha[C]bebit [G|D]hu[Em|G]mus!}

\Repeat{Ubi sunt qui ante nos
in mundo fuere?}
vadite ad superos
transite ad inferos
\Repeat{ubi iam fuere?}

\Repeat{Vita nostra brevis est,
brevi finietur,}
venit mors velociter,
rapit nos atrociter
\Repeat{nemini parcetur!}

\Repeat{Vivat academia,
vivant professores!}
vivat membrum quodlibet,
vivant membra quaelibet,
\Repeat{semper sint in flore!}

\Repeat{Vivant omnes virgines
faciles, formosae,}
vivant et mulieres,
tenerae, amabiles,
\Repeat{bonae, laboriosae}

\Repeat{Vivat et res publica,
et qui illam regit!}
vivat nostra civitas,
maecenatum caritas,
\Repeat{quae nos hic protegit!}

\Repeat{Pereat tristitia
pereant osores}
pereat diabolus
quivis antiburschius
\Repeat{atque irrisores}
\end{guitar}
\end{multicols*}


\chapter*{Landsknechtlieder}
\section*{Traditionell}
\songheader{Unser liebe Fraue vom kalten Bronnen}{um 1536}%%TODO
\begin{guitar}
[Am]Unser liebe [F]Fra[C]ue vom [C]kalten [G]Bron[C]nen,
[C]Bescher uns armen [F]Landsknecht ein' warme Sonnen!
Daß wir nit erfrieren, gehn in des Wirtes Haus
wir ein mit vollem Säckel, mit leerem wieder aus.

\Refraindef Die Drummen, die Drummen, Lärman, lärman, lärman, 
heiridiridiran, ridiran Ride Landsknecht voran.

Unser liebe Fraue vom kalten Bronnen,
bescher' uns armen Landsknecht ein' warme Sonnen!
Daß wir nit erfrieren, zieh'n wir dem Bauermann
das wullen Hemd vom Leibe, das steht ihm übel an.
\Refrain

Wir schlucken Staub beim Wandern der Säckel hängt uns hohl
der Kaiser schluckt ganz Flandern bekomm's ihm ewig wohl
Er denkt beim Länderschmause wie er die Welt erwürb
mir wohnt ein Lieb zu Hause das weinte, wenn ich stürb
\Refrain

Trommler schlägt Parade, die seid'nen Fahnen weh'n,
jetzt heißt's auf Glück und Gnade ins Feld marschieren geh'n.
Korn reift auf den Feldern, es schnappt der Hecht im Strom,
heiß weht der Wind von Geldern, herauf gen Berg op Zoom.
\Refrain

Unser liebe Fraue vom kalten Bronnen,
bescher' uns armen Landsknecht ein' warme Sonnen!
Daß wir endlich finden von aller Arbeit Ruh!
Der Teufel hol das Saufen, das Rauben auch dazu.
\Refrain
\end{guitar}
\songfooter{1536 aufgezeichnet}


\songheader{Oh Magdeburg}{Nach einer Soldatenweise von 1540}%TODO
\begin{multicols*}{2}
\begin{guitar}
O Magdeburg, halt dich feste,
du wohlgebautes Haus!
Dir kommen fremde Gäste,
die woll'n wir treiben aus.

Die Gäst' und die uns kennen seind
Mönch' und Pfaffenknecht',
hilf reicher Christ im Himmel,
daß wird sie grüßen recht.

Zu Magdeburg auf der Brucken,
da liegen drei Hündelein,
sie heulen alle Morgen,
kein'n Spanier lassen sie ein.

Zu Magdeburg steht auf dem Platze
ein großer eiserner Mann,
derselb' nimmt acht der Hatze
und sieht dein'n Spanier an.

Zu Magdeburg auf der Mauern,
da liegt ein gutes Geschütz,
bringt manchem Herzen Trauern,
daß man sie doch nicht nützt.

\columnbreak

Auch liegen an der Zinnen
zwei scharfe Ritterschwert:
Könnten diese Mönch' gewinnen,
wär's manche Knappen wert!

Zu Magdeburg auf der Brucken
sitzen zwei Jungfrauen fein,
die machen alle Morgen
drei Rautenkränzelein.

Das eine soll Herzog Hansen,
dem Fürsten hochgeboren;
Graf Albrecht von Mansfeld
das andre ist erkoren.

Das dritt', das ist versprochen
ein'n Held noch unbekannt,
der läßt nichts ungerochen,
wagt drauf sein Leut und Land.

Dies Liedlein hat gesungen
ein Landsknecht frisch und frei,
stund, da viel Kronen klungen,
daß Gott stets bei ihm sei.
\end{guitar}
\end{multicols*}
%\songfooter{Das Lied hat eine etwas komplexere Melodie als die "Landsknechts-Schlager" der Bündischen Jugend}


\songheader{Die Landsknechte kommen}{}%TODO: Einordnen. Wer hats geschrieben, Zeitangabe?
\begin{guitar}
Die Landsknechte kommen
Trum, trum, terum tum tum
Trum, trum, terum tum tum,
die Landsknecht zieh'n im Land herum.
Trum, trum, terum tum tum,
mit Trommeldröhnen und Gebrumm.
Es schrillen die Flöten,
das Kriegsvolk, es singt,
es flattern die Fahnen,
es jauchzt und es klingt.

Hei, hei, heißa juchei,
die Wallensteiner zieh'n vorbei.
Hei, hei, heißa juchei,
mit Spiel und Feldgeschrei.
Trum, trum, terum tum tum,
Trum, trum, terum tum tum. 
Trum, trum, terum tum tum,
und wieder geht die Trommel um,
trum, trum, terum tum tum,
sie wird nicht müde und nicht stumm.
Sie dreuet dem Schweden im blutigen Krieg,
wir hör'n sie beim Sterben,
wir hör'n sie beim Sieg.
Hei, hei, heißa juchei...

Trum, trum, terum tum tum,
die Trommel trägt die Not herum.
Trum, trum, terum tum tum,
die besten schlägt sie lahm und krumm.
Auf endlosen Wegen, an Trümmern entlang,
bellt fremd uns entgegen der heisere Sang.
Hei, hei, heißa juchei...
\end{guitar}

\songheader{Wir zogen in das Feld}{um 1509}%TODO
\begin{guitar}
Wir zogen in das Feld.
Wir zogen in das Feld.
Da hätten wir weder Säckl noch Geld

Strampede mi 
A la mi presente 
Al vostra signori.

Wir kamen vor Siebentod,
Wir kamen vor Siebentod,
Da hätten wir weder Wein noch Brot.

Strampede mi 
A la mi presente 
Al vostra signori.

Wir kamen vor Friaul,
Wir kamen vor Friaul,
Da hätten wir allesamt voll Maul.

Strampede mi 
A la mi presente 
Al vostra signori.

Wir kamen vor Benevent,
Wir kamen vor Benevent,
Da hat all unser Not ein End.

Strampede mi 
A la mi presente 
Al vostra signori.
\end{guitar}

\newpage\section*{Neuzeitlich}
\songheader{Jörg von Frundsberg, führt uns an!}{Wilhelm Kutschbach, 1934}
\begin{guitar}
[Am]Jörg von Frundsberg, führt uns an,
[Am]Tra la la [Dm]la la la [E]la,
\Repeat{[F]Der die [C/Am]Schlacht [F/Am]gewann,
[E]Lerman vor Pavi-[Am]a.}

Kaiser Franz von Frankenland,
Tra la la la la la la,
Fiel in des Frundsbergs Hand,
Lerman vor Pavia.

Alle Blümlein standen rot,
Tra la la la la la la,
Heißa, wie schneit der Tod,
Lerman vor Pavia.

Als die Nacht am Himmel stund,
Tra la la la la la la,
Trommel und Pfeif' ward kund,
Lerman vor Pavia.

Und der euch dies Liedlein sang,
Tra la la la la la la,
Ward ein Landsknecht genannt,
Lerman vor Pavia.
\end{guitar}

\songheader{Des Geyers schwarzer Haufen}{Anfang 20. Jh.}%TODO: Umbruch
\begin{guitar}
[Am]Wir sind des Geyers schwarzer Haufen - [F]Heyah [Am]Heyoh
[G]Wir wolln mit Pfaff und Adel [Am]raufen - [E]Heyah [Am]Heyoh

\Refraindef [Am]Spieß voran - hey! [F]Rauf und [Am]ran
[G]Setzt aufs Klosterdach den [E]roten [Am]Hahn

[Am]Uns führt der Florian Geyer an, [F]trotz Acht und [Am]Bann,
[G]Den Bundschuh führt er [Am]in der Fahn', hat [E]Helm und Harnisch [Am]an.

\Refrain

[Am]Nun gilt es Schloß Abtei und Stift - [F]Heyah [Am]Heyoh
[G]Uns gilt nichts als die [Am]heil'ge Schrift - [E]Heyah [Am]Heyoh

\Refrain

[Am]Als Adam grub und Eva spann - [F]Kyrie[Am]leis
[G]Wo war denn da der [Am]Edelmann - [E]Kyrie[Am]leis

\Refrain

[Am]Das Reich und der Kaiser hören uns nicht, [F]heia [Am]hoho,
[G]wir halten selber [Am]das Gericht, [E]heia [Am]hoho.

\Refrain

[Am]Ein gleich' Gesetz das wollen wir han', [F]heia [Am]hoho,
[G]vom Fürsten bis [Am]zum Bauersmann, [Em]heia [Am]hoho.

\Refrain

[Am]Wir woll'n nicht länger sein ein Knecht - [F]Heyah [Am]Heyoh
[G]Leibeigen, frönig, [Am]ohne Recht - [E]Heyah [Am]Heyoh

\Refrain

[Am]Bei Weinsberg setzt es Brand und Stank, [F]Heyah [Am]Heyoh
[G]Da mancher über die Klinge [Am]sprang, [E]Heyah [Am]Heyoh

\Refrain

[Am]Sie schlugen uns mit Prügeln platt, [F]heia [Am]hoho,
[G]und machten uns mit [Am]Hunger satt, [E]heia [Am]hoho.

\Refrain

[Am]Geschlagen gehen wir nach Haus - [F]Heyah [Am]Heyoh
[G]Die Enkel fechten's [Am]besser aus - [E]Heyah [Am]Heyoh

\Refrain
\end{guitar}
\songfooter{Der Text des Liedes entstand um 1920 in Kreisen der Jugendbewegung unter Verwendung von Textteilen des Gedichtes "`Ich bin der arme Kunrad"' von Heinrich von Reder (1885), die Melodie stammt von Fritz Sotke (1919) - (Quelle:Wikipedia)}


\songheader{Es klappert der Huf am Stege}{Hans Riedel/Robert Götz, 1920}%TODO
\begin{guitar}
Es klappert der Huf am Stege,
Wir ziehn mit dem Fähnlein ins Feld;
Blut'ger Kampf allerwege,
Dazu sind auch wir bestellt.
Wir reiten und reiten und singen,
Im Herzen die bitterste Not.
Die Sehnsucht will uns bezwingen
Doch wir reiten die Sehnsucht tot. 

Dörfer und Städte flogen
Vorüber an unserem Blick.
Wir sind immer weiter gezogen,
Für uns gibt es kein Zurück.
Wir reiten durch Täler und Hügel,
Wo der Sommer in Blüte steht;
Es knirschen Zaumzeug und Zügel,
Der Wimpel hoch über uns weht. 

Leis sinkt der Abend nieder,
Uns wird das Herz so schwer;
Leiser werden die Lieder,
Wir sehn keine Heimat mehr.
Wir reiten und reiten und reiten
Und hören von fern schon die Schlacht,
Herr, laß uns stark sein im Streiten,
Dann sei unser Leben vollbracht.
\end{guitar}


\songheader{Wilde Gesellen}{Fritz Sotke, 1923}%TODO
\begin{guitar}
Wilde Gesellen vom Sturmwind durchweht,
Fürsten in Lumpen und Loden,
Ziehn wir dahin bis das Herze uns steht,
Ehrlos bis unter den Boden.
Fidel Gewand in farbiger Pracht
Trefft keinen Zeisig ihr bunter,
Ob uns auch Speier und Spötter verlacht,
Uns geht die Sonne nicht unter.

Ziehn wir dahin durch Braus und durch Brand,
Klopfen bei Veit und Velten.
Huldiges Herze und helfende Hand
Sind ja so selten, so selten.
Weiter uns wirbelnd auf staubiger Straß
Immer nur hurtig und munter;
Ob uns der eigene Bruder vergaß,
Uns geht die Sonne nicht unter.

Aber da draußen am Wegesrand,
Dort bei dem König der Dornen.
Klingen die Fiedeln ins weite Land,
Klagen dem Herrn unser Carmen.
Und der Gekrönte sendet im Tau
Tröstende Tränen herunter.
Fort geht die Fahrt durch den wilden Verhau,
Uns geht die Sonne nicht unter.

Bleibt auch dereinst das Herz uns stehn
Niemand wird Tränen uns weinen.
Leis wird der Sturmwind sein Klagelied wehn
Trüber die Sonne wird scheinen.
Aus ist ein Leben voll farbiger Pracht,
Zügellos drüber und drunter.
Speier und Spötter, ihr habt uns verlacht,
Uns geht die Sonne nicht unter.
\end{guitar}

\chapter*{Folk (International)}
\begin{landscape}
\songheader{Young Ned of the Hill (The Pogues)}{\vspace{-3\baselineskip}}%%TODO: Layout
\begin{multicols*}{2}
\begin{guitar}
[Am]Have you ever walked the [Em]lonesome hills
[Am]And heard the curlews [Em]cry
[Am]Or seen the raven [G]black as night
[F]Upon a windswept [Am]sky
[Am]To walk the purple heather
[Am]And hear the westwind [Em]cry
[Am]To know that's where the [G]rapparee must [Am]die
[Am] \ [G] \ [Em] \ [D] \ [C] \ [Am] \ [C] \ [Dm] \ [Am]

Since Cromwell pushed us westward
To live our lowly lives
Some of us have deemed to fight
From Tipperary mountains high
Noble men with wills of iron
Who are not afraid to die
Who'll fight with gaelic honour held on high

\Refraindef
A curse upon you Oliver Cromwell
You who raped our Motherland
I hope you're rotting down in hell
For the horrors that you sent
To our misfortunate forefathers
Who ye robbed of their birthright
To hell or Connaught, may you burn in hell tonight

Of one such man I'd like to speak
A rapparee by name and deed
His fam'ly dispossessed \& slaughtered
They put a price upon his head
His name's known in song \& story
His deeds are legends still
And murdered for blood money
Was young Ned of the hill

\Refrain

You have robbed our homes \& fortunes
Even drove us from our land
You tried to break our spirit
But you'll never understand
The love of dear old Ireland
That will forge an iron will
As long as there are gallant men
Like young Ned of the hill
\end{guitar}
\end{multicols*}
\end{landscape}

\songheader{Ye Jacobites by Name}{Robert Burns, 1791}
\begin{guitar}
[Am]Ye Jacobites by name, [C]lend an ear, [G]lend an ear,
[Am]Ye Jacobites by [Em]name, lend [(G)]an [Am]ear,
[C]Ye Jacobites by name,
[G]Your faults I will proclaim,
\Repeat{[Am]Your doctrines I must [Em]blame, (ye [C]shall hear), ye [Em]shall hear}

What is Right, and what is Wrong, by the law, by the law?
What is Right and what is Wrong by the law?
What is Right, and what is Wrong?
A short sword or a long,
A weak arm or a strong, for to draw.

What makes heroic strife, famed afar, famed afar?
What makes heroic strife famed afar?
What makes heroic strife?
To whet the assassin's knife,
Or hunt a Parent's life, in bloody war?

Then let your schemes alone, in the state, in the state,
Then let your schemes alone in the state.
Then let your schemes alone,
Adore the rising sun,
And leave a man undone, to his fate.
\end{guitar}


\chapter*{Volksliedgut und Vagantenlieder}
\songheader{Papst und Sultan}{Text: Christian Ludwig Noack (1767-1821), Melodie: Zupfgeigenhansel}
\begin{multicols*}{2}
\begin{guitar}
Der [C]Papst lebt herrlich in der [G]Welt,
es [G]fehlt ihm nie an Ablaß[C]geld;
[C]er trinkt vom aller[G]besten [Am]Wein:
drum möcht' ich [F]auch d. [G]P. wohl [C]sein.

Doch nein, er ist ein armer Wicht,
ein holdes Mädchen küßt ihn nicht;
er schläft in seinem Bett allein:
drum möchte ich der Papst nicht sein.

Der Sultan lebt in Saus und Braus,
er wohnt in einem Freudenhaus
voll wunderschönen Mägdelein:
drum möcht ich wohl der Sultan sein.

Doch nein, er ist ein armer Mann,
denn folgt er seinem Alkoran,
so trinkt er keinen Tropfen Wein:
drum möcht ich auch $\neg$ Sultan sein.

Geteilt veracht ich beider Glück
und kehr in meinen Stand zurück;
doch das geh ich mit Freuden ein:
halb Sultan und halb Papst zu sein.

Drum Mädchen, gib mir einen Kuß,
denn jetzt bin ich dein Sultanus!
Ihr trauten Brüder, schenket ein,
damit ich auch der Papst kann sein!

\columnbreak
Pa[C]pa est in lautit[G]iis
Cum [G]poenitentiae nummu[C]lis,
[C]Haurit Fa[G]lerni pocu[Am]la:
Quam velim [F]ipse [G]sim pa[C]pa!

At miser est homunculus,
Caret uxore lectulus,
Quae nectat ei brachia.
Quam nolim ipse sim papa!

Luxuriosus sultanus
Auratis agit lectibus,
Plena puellarum domo,
Velim sultanus sim ego!

At homo est miserrimus
Corano parens sultanus!
Non bibit vini guttulam.
Auferte vitam Turcicam!

Non minus est optanda mors,
Quam separata horum sors.
At id accipio statim,
Ut papa nunc, nunc Turcus sim.

Puellula, da osculum!
Nam ecce, sultanus nunc sum!
Implete, fratres, pocula,
Ut iterum tunc sim papa!
\end{guitar}
\end{multicols*}

\songheader{Es dunkelt schon auf der Heide}{\todo}
\begin{guitar}
Es [A]dunkelt schon [D]in der [A]Heide, nach [A]Hause [E]lass uns [A]geh'n.
\Repeat{[A]Wir haben das Korn geschnitten mit [D]unserm [E]blanken [A]Schwert.}

Ich hörte die Sichel rauschen, ja rauschen durch das Korn.
Ich hörte mein Feinslieb klagen, sie hätte ihr Lieb' verlor'n.

Hast du dein Lieb' verloren, so hab ich noch das mein;
so wollen wir beide mit'nander uns winden ein Kränzelein.

Ein Kränzelein von Rosen, ein Sträußelein von Klee,
zu Frankfurt an der Brücke, da liegt ein tiefer Schnee.

Der Schnee, der ist zerschmolzen, das Wasser läuft dahin,
kommst mir aus meinem Auge, kommst mir nicht aus dem Sinn.

In meines Vaters Garten, da steh'n zwei Bäumelein;
das eine trägt Muskaten, das and're Braunnägelein.

Muskaten, die sind süße, Braunnägelein sind schön;
wir beide uns müssen scheiden, ja scheiden, das tut weh.
\end{guitar}


\songheader{Es wollt' ein Bauer früh aufstehn}{\todo}
\begin{multicols*}{2}
\begin{guitar}
[C]Es wollt' ein Bauer früh aufstehn,
[C]Es wollt' ein Bauer [G]früh aufstehn,
[G]Und hinaus auf seinen [C]Acker gehn
[G]Fallera--di--[C]rallala, [G]Fallera-la-[C]la

Und als er dann nach Hause kam
Da wollt er was zu Fressen ham

"`Ei Lieschen koch mir Hirsebrei
Mit Bratkartoffeln und Spiegelei!"'

Und als der Bauer saß und aß
Da rumpelt in der Kammer was

"`Ei, liebe Frau was ist denn das?
Da rumpelt in der Kammer was"'

"`Ei, lieber Mann das ist der Wind
Der raschelt da am Küchenspind"'

Der Bauer sprach:"`Will selber sehn
Will selber raus in die Kammer gehn"'

\columnbreak

Und als der Bauer in d'Kammer kam
Da zog der Pfaff die Hosen an

"`Ei, Pfaff was machst in meinem Haus
Ich werf dich ja sogleich hinaus!"'

Der Pfaff der sprach: "`Was ich verricht
Deine Frau, die kennt die Beicht noch nicht"'

Da nahm der Baur ein Ofenscheit
Und schlug den Pfaffen daß er schreit

Der Pfaff der schrie: "`Oh Schreck, oh Graus!"'
Und hielt den Arsch zum Fenster raus

Da kam' die Leut von nah und fern
Und dachten's wär der Morgenstern

Der Morgenstern der war es nicht
Es war des Pfaffen Arschgesicht

Und die Moral von der Geschicht
Trau nicht des Pfaffen Arschgesicht
\end{guitar}
\end{multicols*}


\songheader{Ein Mönch kam vor ein Nonnenkloster}{Trad., ca. 17. Jhd.}
\begin{multicols*}{2}
\begin{guitar}
Ein [C]Mönch kam vor 1 Nonnenkloster
-- [C]Ei, [G]ei, [C]ei!
Mit [C]seinem langen [G]Paternoster,
-- [G]fallahi und [C]fallaha!
Mit [C]seinem langen [G]Paternoster,
-- [G]fallahi-a-[C]he!
\Repeat{Und seinem
[C/G]Kling klang klonimus [G/C]Dominus}
[F]Orati[G]oni[C]mus

Der Pater klopft an die Klostertür
Da schaut ne kranke Nonn' herfür
Für seinen ...

Der Pater steigt die Trepp' hinauf
Die Nonne schaut von unten auf
Sie sah sein ...

"`Ei Pater was'n das für'n Ding?
Was unter eurer Kutten schwingt"'
Ist das ein ...

\columnbreak

"`Ja das ist mein Patientenstab
Mit dem ich kranke Nonnen lab"'
Das ist mein ...

"`\Repeat{Ei Pater}, so labet auch mich!
Die kränkste aller Nonn'n bin ich!"'
Mit eurem ...

Er ging mit ihr auf den Orgelboden
Und orgelte nach allen Noten
Mit seinem ...

"`Ei Pater das hat wohlgetan!"'
"`Da fang'n wir gleich von vorne an!"'
Sieh nur mein ...

Und die Moral von der Geschicht
Wer's nicht versteht, der orgele nicht!
Mit einem ...
\end{guitar}
\end{multicols*}


\songheader{Bunt sind schon die Wälder}{\todo}
\begin{multicols*}{2}
\begin{guitar}
[G]Bunt sind schon die [D]Wäl[G]der,
[G]Gelb die Stoppel[C]fel[G]der,
[D]Und der [A]Herbst be[D]ginnt.
[G]Rote Blätter [C]fallen[E],
[Am]Graue Nebel [D]wallen, 
[G]Kühler [D]weht der [G]Wind.
%[C]Bunt sind schon die [G]Wäl[C]der,
%[C]Gelb die Stoppel[F]fel[C]der,
%[G]Und der [C]Herbst be[G]ginnt.
%[C]Rote Blätter [Am]fallen,
%[F]Graue Nebel [G]wallen, 
%[C]Kühler [G]weht der [C]Wind.

Wie die volle Traube
Aus dem Rebenlaube
Purpurfarbig strahlt!
Am Geländer reifen
Pfirsiche, mit Streifen
Rot und weiß bemalt.

Dort im grünen Baume
hängt die blaue Pflaume
am gebognen Ast.
Gelbe Birnen winken
dass die Zweige sinken
unter ihrer Last.

Welch ein Äpfelregen
rauscht vom Baum. Es legen
in ihr Körbchen sie
Mädchen leicht geschürzet
und ihr Röckchen kürzet
sich bis an das Knie.

\columnbreak

Winzer, füllt die Fässer!
Eimer, krumme Messer,
Butten sind bereit!
Lohn für Müh' und Plage
sind die frohen Tage
in der Lesezeit!

Unsre Mädchen singen
und die Träger springen,
alles ist so froh.
Bunte Bänder schweben
zwischen hohen Reben
auf dem Hut von Stroh.

Flinke Träger springen,
Und die Mädchen singen,
Alles jubelt froh!
Bunte Bänder schweben
Zwischen hohen Reben
Auf dem Hut von Stroh.

Geige tönt und Flöte
Bei der Abendröte
Und im Mondesglanz;
Junge Winzerinnen
Winken und beginnen
Frohen Erntetanz.
\end{guitar}
\end{multicols*}



\songheader{Wenn alle Brünnlein fließen}{\todo}
\begin{guitar}
Wenn [D]alle [A]Brünnlein [D]fließen,
So [G]muß man [Em]trin-[A]ken
Wenn [D]ich mein'n [A]Schatz nicht [D]rufen darf,
Tu [G]ich ihm [Em]win-[A]ken,
Wenn [A]ich mein'n Schatz nicht [D]rufen darf,
[A]Ju, ja, [D]rufen darf,
Tu ich ihm [A]win-[D]ken.

Ja, winken mit den Äugelein,
Und treten auf den Fuß;
's ist eine in der Stube drin,
Die meine werden muß,
's ist eine in der Stube drin,
Ju, ja, Stube drin,
Die meine werden muß,

Warum sollt sie's nicht werden,
Ich hab' sie ja so gern;
Sie hat zwei blaue Äugelein,
Die leuchten wie zwei Stern,
Sie hat zwei blaue Äugelein,
Ju, ja, Äugelein,
Die leuchten wie zwei Stern.

Sie hat zwei rote Wängelein,
Sind röter als der Wein;
Ein solches Mädel findst du nicht
Wohl unterm Sonnenschein;
Ein solches Mädel findst du nicht,
Ju, ja, findst du nicht,
Wohl unterm Sonnenschein.
\end{guitar}

\songheader{Zogen einst fünf wilde Schwäne}{\todo}
\begin{guitar}
{\tiny (Jede Zeile wird zweimal gesungen. Das eingeklammerte "`ja"' wird nur beim ersten Mal gesungen.)}
Es [A]zogen [D]einst [A]fünf wilde [F#m]Schwäne, [D]Schwäne leuchtend [E]weiß und [A]schön.
[E]Sing, sing, [A]was geschah? [E]Keiner ward mehr [A]gesehen(, ja).

Es zogen einst fünf junge Burschen stolz und kühn zum Kampf hinaus.
Sing, sing, was geschah? Keiner kam mehr nach Haus(, ja).

Es wuchsen einst fünf junge Birken, schlank und grün an Bachesrand.
Sing, sing, was geschah! Keine in Blüte stand(, ja).

Es wuchsen einst fünf junge Mädchen schlank und schön am Memelstrand.
Sing, sing, was geschah? Keine den Brautkranz wand(, ja).

\Repeat{Sing, sing, was geschah? Keine den Brautkranz wand(, ja).}
\end{guitar}


\songheader{Es, es, es und es}{\todo}
\begin{multicols*}{2}
\begin{guitar}
[G]Es, [D]es, [G]es und es,
Es ist ein [D]harter [G]Schluss,
Weil, [D]weil, [G]weil und weil,
weil ich aus [D]Frankfurt [G]muss!
Drum schlag ich F. [D]aus dem Sinn
Und [G]wende mich Gott [A]weiß [D]wohin.
Ich [G]will mein [D]Glück [G]probieren,
Marsch-[D]ier[G]-en.

Er, er, er und er,
Herr Meister, leb er wohl!
Er, er, er und er,
Herr Meister, leb er wohl!
Ich sag's ihm grad frei ins Gesicht,
Seine Arbeit \& sein Lohn gefällt mir nicht.
Ich will mein Glück probieren,
Marschieren.

Sie, sie, sie und sie,
Frau Meistrin leb sie wohl!
Sie, sie, sie und sie,
Frau Meistrin leb sie wohl!
Ihr Essen war so angericht',
Das fraßen selbst die Schweine nicht
Ich will mein Glück probieren,
Marschieren.

\columnbreak

Er, er, er und er,
Herr Wirt, nun leb er wohl!
Er, er, er und er,
Herr Wirt, nun leb er wohl!
Hätt er die Kreid nicht doppelt geschrieben,
So wär ich noch länger dageblieben
Ich will mein Glück probieren,
Marschieren.

Ihr, ihr, ihr und ihr,
Ihr Brüder lebet wohl!
Ihr, ihr, ihr und ihr,
Ihr Brüder lebet wohl!
Hab ich euch was zuleid getan
So bitt' ich um Verzeihung an.
\Repeat{Ich will mein Glück probieren,
Marschieren.}
\end{guitar}
\end{multicols*}


\chapter*{Moderne Balladen}
\songheader{Andre, die das Land so sehr nicht liebten}{Text: Theodor Kramer \quad Melodie: E. Schmeckenbecher (Zupfgeigenhansel)}
\begin{guitar}
[C]Andre, die das Land so sehr nicht [G]liebten,
[Am]warn von Anfang [F]an gewillt zu [C]gehn' [G];
[Am]ihnen - manche [F]sind schon fort - [C]ist besser,
[C]ich doch müßte mit dem eignen [G]Messer
[Am/F]meine Wurzeln [G]aus der Erde [C]drehn.

Keine Nacht hab ich seither geschlafen,
und es ist mir mehr als weh zumut;
viele Wochen sind seither verstrichen,
alle Kraft ist längst aus mir gewichen,
und ich fühl, daß ich daran verblut.

Und doch müßt ich mich von hinnen heben,
sei's auch nur zu bleiben, was ich war.
Nimmer kann ich, wo ich bin, gedeihen;
draußen braucht ich wahrlich nicht zu schreien,
denn mein leises Wort war immer wahr.

Seiner wär ich wie in alten Tagen
sicher; schluchzend wider mich gewandt,
hätt ich Tag und Nacht mich nur zu heißen,
mich samt meinen Wurzeln auszureißen
und zu setzen in ein andres Land.
\end{guitar}

\songheader{Mein Michel, was willst du noch mehr?}{Text: um 1918 \quad Melodie: Zupfgeigenhansel}
\begin{guitar}
1. [Am]Du hast [G]Batallionen, [Am]Schwadronen
Batter[C]ien, Ma[G]schinenge[C]wehr
du [Dm]hast auch die größten [C]Kano[Dm]nen
\Repeat{[Dm]Mein [Am]Michel, was [Dm/G]willst du noch [Am]mehr}

2. Du hast zwei dutzend Monarchen,
Lakaien und Pfaffen, ein Heer.
Da kannst du beseligt schnarchen,
Mein Michel was willst du noch mehr...

3. Du hast ungezählt Paragraphen,
deine Gefängnisse werden nicht leer.
da kannst du in Schutzhaft drin schlafen,
Mein Michel was willst du noch mehr...

4. Du zahlst die beträchtlichsten Steuern,
deine Junker plagen sich sehr,
um dir das Brot zu verteuern,
Mein Michel was willst du noch mehr...

5. Du hast Kohlrüben und Eicheln,
und trägst du nach anderm Begehr,
so kannst du den Bauch dich streicheln,
Mein Michel was willst du noch mehr...

6. Du darfst exerzieren, marschieren,
am Kasernenhof, kreuz und die quer,
und dann für den Kaiser krepieren,
Mein Michel was willst du noch mehr...
\end{guitar}

\begin{landscape}
\songheader{Mit Wunder jetzunder}{Trad. aus :"`Ebermannstädter Liederhandschrift"', 1750}
\begin{multicols*}{2}
\begin{guitar}
[C]Mit Wunder [G]jezunder man [C]sehen kann [G]recht
[F]wie mancher [C]verachtet das [G]Baurenge[C]schlecht
[C]er bildet ihm ein, [C]viel besser zu sein
[C]als Bauren, die bauen [d]das Korn und den [G]Wein
[C]Wer will nun jetzt [G]zweifeln, wo [C]solches kommt [G]her
[F]das Kisten und [C]Kasten und [G]Scheuren so [C]leer.

Wann aber ein jeder die Sache bedächt
wie das wir seynd alle von Baurengeschlecht
auch kommen seynd her, so würde nun Er
den Bauren auch geben gebührende Ehr
Der Kayser, der König, der Bürger im Land
sich müssen ernähren von's Bauren sein Hand.

Ja! wann man thut hören vom Kriegesgeschrey
wo nimmt man her Haber? wo nimmt mann her Heu?
wo nimmt man her Holz? wo nimmt man her Schmalz?
die Bauren, die müssen herschaffen ja all's
Wann d'Herren sich raufen, sag: ist es nicht wahr?
So müssen die Bauren hergeben die Haar.

Wie mancher verachtet die Bauren auf Erd
vor alters Gott selbsten die Bauren hielt wert
wie solches man dann schön lesen noch kann
in heiligen Schriften: drum höret mich an
der Elisaeus auch ein Bauersmann war
Gott macht ihn jedoch zum Propheten so gar.

Hat Gott der Herr selbsten kein Bauren veracht
aus Bauren berühmte Propheten gemacht
drum jeder denk fein, und bilde ihm ein
das Er nicht woll besser als Bauersleut seyn
die Bauren auf Erden seynd Ehrens wohl wert
die--weil sich ein jeder vom Bauren ernährt.
\end{guitar}
\end{multicols*}
\end{landscape}

\songheader{Wer jetzig Zeiten leben will}{}
\begin{multicols*}{2}
\begin{guitar}
Wer [G]jetzig Zeiten [C]leben [D]will, 
muss [Am]han ein [D]tapfer [G]Herze.
Er hat der argen [C]Feind so [D]viel, 
be-[Am]reiten [D]ihm groß [G]Schmerze.

Da [G]heißt es steh'n ganz [Em]unver-[Am]zagt 
in [G]seiner blanken [D]Wehre,
dass [G]sich der Feind nicht [C]an uns [D]wagt, 
es [Am]geht um [D]Gut und [G]Ehre.

Geld nur regiert die ganze Welt, 
dazu verhilft Betrügen,
wer sich sonst noch so redlich hält, 
muss doch bald unterliegen.

Rechtschaffen hin, rechtschaffen her, 
das sind nur alte Geigen:
Betrug, Gewalt und List vielmehr, 
klag du, man wird dir's zeigen.

\columnbreak

Der Kipper, Teufel und Soldat, 
die haben itzt Gewalten,
was sonst noch ist, kein Ansehn hat - 
wie soll man Recht behalten?

Weg da Gesetz, weg da mit Recht, 
die können nichts entscheiden.
Der Klein' ist so des Großen Knecht, 
so alles muss erleiden.

Doch wie's auch kommt, das arge Spiel: 
Behalt ein tapfer Herze,
und sind der Feind' auch noch so viel: 
Verzage nicht im Schmerze!

Steh selbstbewusst und unverzagt 
in deiner blanken Wehre:
Wenn sich der Feind jetzt an uns wagt, 
es geht um Gut und Ehre!
\end{guitar}
\end{multicols*}

\songheader{Die Gedanken sind frei}{}
\begin{guitar}
Die [G]Gedanken sind frei, wer [D]kann sie [G]erraten,
sie eilen vorbei, wie [D]nächtliche [G]Schatten.
Kein [D]Mensch kann sie [G]wissen, kein [D]Jäger er-[G]schießen.
Mit [C]Pulver oder [G]Blei: Die [D]Gedanken sind [G]frei!

Ich denke was ich will und was mich beglückt,
doch alles in der Still', und wie es sich schicket.
Mein Wunsch und Begehren kann niemand verwehren,
es bleibet dabei: Die Gedanken sind frei!

Ich liebe den Wein, mein Mädchen vor allen,
sie tut mir allein am besten gefallen.
Ich bin nicht alleine bei einem Glas Weine,
mein Mädchen dabei: Die Gedanken sind frei!

Und sperrt man mich ein in finsteren Kerker,
das alles sind rein vergebliche Werke.
Denn meine Gedanken zerreißen die Schranken
und Mauern entzwei, die Gedanken sind frei!

Drum will ich auf immer den Sorgen entsagen
und will mich auch nimmer mit Grillen mehr plagen.
Man kann ja im Herzen stets lachen und scherzen
und denken dabei: Die Gedanken sind frei!
\end{guitar}

\songheader{Schließ Aug' und Ohr für eine Weil}{}
\begin{guitar}
Schließ [Dm]Aug' und Ohr für [A]eine Weil
vor [Gm]dem Getös [A]der [Dm]Zeit.
Du [F]heilst es nicht und [C]hast kein [A]Heil
als [Dm]wo dein [A]Herz sich [Dm]weiht. 

Dein Amt ist hüten, harren, sehn
im Tag die Ewigkeit.
Du bist schon so im Weltgeschehen
befangen und befreit.

Die Stunde kommt da man dich braucht.
Dann sei du ganz bereit.
Und in das Feuer, das verraucht,
wirf dich als letztes Scheit.

Dein Amt ist Hüten, Harren, Sehen
In die Ewigkeit.
So bist du schon im Weltgeschehen
Befangen und befreit.
\end{guitar}




\songheader{Das weiche Wasser}{Dieter Dehm}
\begin{guitar}
Europa hatte zweimal Krieg
der dritte wird der letzte sein.
Gib bloß nicht auf, gib nicht klein bei,
das weiche Wasser bricht den Stein.


Die Bombe, die kein Leben schont,
Maschinen nur und Stahlbeton.
Hat uns zu einem Lied vereint
das weiche Wasser bricht den Stein.


Es reißt die schwersten Mauern ein
und sind wir schwach und sind wir klein,
wir wollen wie das Wasser sein,
das weiche Wasser bricht den Stein.


Raketen steh'n vor uns'rer Tür,
die soll'n zu uns'rem Schutz hier sein.
Auf solchen Schutz verzichten wir,
das weiche Wasser bricht den Stein.


Es reißt die schwersten Mauern ein
und sind wir schwach und sind wir klein,
wir wollen wie das Wasser sein,
das weiche Wasser bricht den Stein.


Die Rüstung sitzt am Tisch der Welt,
und Kinder, die vor Hunger schrei'n,
für Waffen fließt das große Geld,
das weiche Wasser bricht den Stein.


Komm feiern wir ein Friedensfest,
und zeigen, wie sich's leben läßt.
Mensch, Menschen können Menschen sein,
das weiche Wasser bricht den Stein.

{\tiny "`Das weiche Wasser bricht den Stein"' wurde Ende der 70er / Anfang der 80er Jahre von Dieter Dehm für die Friedensbewegung geschrieben, friedensbewegter Barde der SPD.}
\end{guitar}

\songheader{Zehn Orks}{Text: Die Streuner, Melodie: Trad. jiddisch}
\begin{multicols*}{2}
\begin{guitar}
Zehn Orks sennen wir gewesen
Haben wir nit geblieben daheim
Haben Elfenstamm getroffen
Sennen wir geblieben neun

Neun Orks sennen wir gewesen
Haben wir über Zwerg gelacht
Zwerg der hat nicht mitgelacht
Sennen wir geblieben acht

\Refraindef
Grishnak mit große Hammer
Zorg mit rostig Schwert
Raufen, Saufen, Schnaufen
Bis sich nichts mehr wehrt ...

Acht Orks sennen wir gewesen
Wollten wir Menschenweiber lieben
Eine sennen Mann gewesen
Sennen wir geblieben sieben

Sieben Orks sennen wir gewesen
Wollten wir reiten auf die Besen von die Hex'
Mauer dort im Weg gestanden
Sennen wir geblieben sechs
\Refrain

\columnbreak

Sechs Orks sennen wir gewesen
Ham' wir ausgezogen unsere Strimpf'
Einer nit beliftet war
Sennen wir geblieben finf

Finf Orks sennen wir gewesen
Haben wir getrunken eigen Bier
Eins war mit Glykol gepanscht
Sennen wir geblieben vier
\Refrain

Vier Orks sennen wir gewesen
Haben wir gemacht d' Bäume entzwei
Baum san leider Ent gewesen
Sennen wir geblieben drei

Drei Orks sennen wir gewesen
Haben wir gefunden scheenes Ei
Mütterchen Drach's vorbeigekommen
Sennen wir geblieben zwei
\Refrain

Zwei Orks sennen wir gewesen
Hatten wir zu Essen leider keiner
Als Briederchen geschlafen hat
Bin ich mir geblieben einer

Ein Ork bin ich mir gewesen
Wollt' ich andere Orks zurick
Seitdem bin ich auf der Suche
Nach einen guten F...rau
\Refrain
\end{guitar}
\end{multicols*}



\chapter*{Seemannslieder}
\songheader{Alle, die mit uns auf Kaperfahrt fahren}{}
\begin{guitar}
\Repeat{[Am]Alle, die mit uns auf Kaperfahrt fahren
[Am]Müssen Männer mit Bärten sein}
[Am]Jan und Hein und [G]Klaas und [C]Pit - [Am]die haben [G]Bärte, die haben [Am]Bärte
[Am]Jan und Hein und [G]Klaas und [C]Pit - [Am]die haben Bärte, die [Em]fahren [Am]mit

Alle, die Weiber und Branntwein lieben \ldots

Alle, die mit uns das Walroß töten \ldots

Alle, die Tod und Teufel nicht fürchten \ldots

Alle, die öligen Zwieback lieben \ldots

Alle, die endlich zur Hölle mitfahren \ldots

Alle die Frauen und Branntwein lieben \ldots
\end{guitar}

\songheader{Hamburger Veermaster}{\todo}
\begin{guitar}
Ick heff mol en Hamborger Veermaster sehn, 
To my ho dae! To my ho dae! 
De Masten so scheef as den Schipper sien Been, 
To my ho dae ho dae ho ho ho ho! 
\Refraindef Blow boys blow for Californio, 
There is plenty of Gold 
So I've been told, 
On the banks of Sacramento. 

Dat Deck weur vun Isen, Vull Schiet uns vull Schmeer. 
To my ho dae! To my ho dae!
"`Rein Schipp"' weur den Käpten Sin grötstet Pläseer.
To my ho dae ho dae ho ho ho ho!
\Refrain

Dat Logis weur vull Wanzen, De Kombüs weur vull Dreck, 
To my ho dae, to my ho dae! 
De Beschüten, de leupen Von sülben all weg. 
To my ho dae ho dae ho ho ho ho! 
\Refrain

Dat Soltfleesch weur greun, Un de Speck weur vull Moden. 
To my ho dae, to my ho dae! 
Köm gäv dat bloß an Wiehnachtsobend. 
To my ho dae ho dae ho ho ho ho! 
\Refrain

Un wulln wi mol seiln, Ick segg dat jo nur, 
To my ho dae, to my ho dae! 
Denn leup he dree vorut Und veer wedder retur. 
To my ho dae ho dae ho ho ho ho! 
\Refrain

As dat Schipp weur so weur Ok de Kaptein, 
To my ho dae, to my ho dae! 
De Lüd for dat Schipp weurn Ok blot schangheit. 
To my ho dae ho dae ho ho ho ho! 
\Refrain
\end{guitar}

\songheader{Spanish Ladies}{}
\begin{guitar}
[Am]Farewell and adieu to [C]you, Spanish [Em]Ladies,
[Am]Farewell and adieu to you, [C]ladies of [Em]Spain;
[C]For we've received orders for to [Am]sail for old [Em]England,
But we [Am]hope in a short time to [Em]see you [Am]again.

\Refraindef We will rant and we'll roar like true British sailors,
We'll rant and we'll roar all on the salt sea.
Until we strike soundings in the channel of old England;
From Ushant to Scilly is thirty five leagues.

We hove our ship to with the wind from sou'west, boys
We hove our ship to, deep soundings to take;
'Twas forty-five fathoms, with a white sandy bottom,
So we squared our main yard and up channel did make.
\Refrain

The first land we sighted was called the Dodman,
Next Rame Head off Plymouth, Start, Portland and Wight;
We sailed by Beachy, by Fairlight and Dover,
And then we bore up for the South Foreland light.
\Refrain

Then the signal was made for the grand fleet to anchor,
And all in the Downs that night for to lie;
Let go your shank painter, let go your cat stopper!
Haul up your clewgarnets, let tacks and sheets fly!
\Refrain

Now let ev'ry man drink off his full bumper,
And let ev'ry man drink off his full glass;
We'll drink and be jolly and drown melancholy,
And here's to the health of each true-hearted lass.
\Refrain
\end{guitar}

\songheader{Arthur McBride}{Trad. irish / Paul Brady / Bob Dylan}
\begin{guitar}
Oh me[G] and my cousin one A[G]rthur McBride
As we[C] went a-wal[G]king down by[Am7] the seasi[C]de
A-ma[G]rking what followed and what[G] might betide
For it being on Christmas mo[D]rning
And f[G]or recreation we we[G]nt on a tramp
And we met[C] Sergeant Har[G]per and Cor[Am7]poral [C]Ramp
And the li[G]ttle wee drummer intending to camp
For the day being pleasant and cha[D]rming[G].

"`Good morning, good morning"' the Sergeant he cried,
"`And the same to you gentlemen"' we did reply,
Intending no harm as we meant to pass by,
For it being on Christmas morning.
But says he "`My fine fellows if you will enlist
It's ten guineas in gold I will slip in your fists.
And a crown in the bargain for to  kick up the dust
And drink the King's health in the morning.

For a soldier he leads a very fine life,
He always is blessed with a charming young wife,
And he pays all his debts without sorrow and strife,
And he always lives pleasant and charming.
And a soldier he always is decent and clean,
In the finest of clothing he's constantly seen.
While other poor fellows look dirty and mean,
And sup on thin gruel in the morning"'

But says Arthur "`I wouldn't be proud of your clothes,
For you've only the lend of them, as I suppose.
And you dare not change them one night for you know,
If you do you'll be flogged in the morning.
And although that we are single and free,
We take great delight in our own company,
And we have no desire strange faces to see,
Although that your offers are charming.
And we have no desire to take your advance,
All hazards and dangers we barter on chance.
For you would have no scruple for to send us to France,
Where we would get shot without warning.

"`Oh no,"' says the Sergeant, "`I'll hear no such chat,
And I never will take it from spalpeen or brat.
For if you insult me with one other word,
I'll cut off your heads in the morning."'
And then Arthur and I we soon drew our odds,
And we scarce gave them time for to draw their own blades,
When a trusty shillelagh came over their heads,
And bade them take that as fair warning.

And their old rusty rapiers that hung by their sides,
We flung them as far as we could in the tide.
"`Now take them out, devils,"' cried Arthur McBride,
"`And temper their edge in the morning"'
And the little wee drummer we flattened his pouch,
And we made a foot-bowl of his rowdy-dowd-dowd,
Threw it in the tide for to rock and to roll,
And bade it a tedious returning.

And we having no money, paid them off in cracks,
And we paid no respect to their two bloody backs.
But we lathered them there like a pair of wet sacks,
And left them for dead in the morning.
And so to conclude and to finish disputes,
We obligingly asked if they wanted recruits,
For we were the lads who would give them hard clouts,
And bid them look sharp in the morning.
\end{guitar}

\songheader{Ich bin ein freier Bauernknecht}{}
\begin{multicols*}{2}
\begin{guitar}
[Gm]Ich bin ein freier [Cm]Bauern[B]knecht
ob [Eb]mein Stand gleich [F]ist eben [B]schlecht
So [B]hab ich [F]mich [B]doch ebenso[D] gut
Als [Gm]einer [D]der am [Gm]Hofe [D]tut
[g]Trallti[D]ralla
Ich [Gm]bin doch mein [Cm]eigen
[Gm]Darf mich vor keinem 
[Cm]bücken noch [D]nei-[Gm]gen

Trag ich gleich keinen Biberhut
So ist ein rauer Filz mir gut
Darauf ein grünen Busch gelegt,
So wohl als teure Federn steht
Tralltiralla
Ich tu es nicht achten 
Obschon die Hofleut spöttisch drauf lachen. 

Trag ich nicht lange krause Haar 
Und Pulver drein, das Geld ich spar 
Der Staub vom Lande weht der Wind
Des Sommers in mein Haar g'schwind
Traltiralla
Drum geh ich gestutzet 
Obschon mein Haar ist vorn geputzet

\columnbreak

Ich habe auch keinen Rittersitz
Bin nicht beredt, voll List und Witz
So hab ich doch ein Bauerngut
Bin frisch fröhlich doch von Mut
Traltiralla
Bin drauf beflissen
Was einem Bauern dient zu wissen.

Ich bin gar selten krank von Leib 
Das macht, dass ich d. Pflug oft treib
Jener aber säuft und frist
Das macht, dass er so krank oft ist
Traltiralla
Bin frischer daneben
als jene, die am Hofe stets leben.

Was bildet sich der Hofmann ein
Dass er als ich will besser sein
Da Adam ackert und Eva spann 
Wer war damals ein Edelmann
Taltiralla
Ich leb alle Morgen 
Sicher und frei von allen Sorgen.
\end{guitar}
\end{multicols*}


\songheader{Jeder Traum}{Text: Louis Fürnberg \quad Musik: Pit Budde (Cochise)}
\begin{guitar}
[G4]Jeder [G]Traum, an den ich mich verschwendet, 
Jeder Kampf, wo ich mich nicht geschont, 
Jeder Sonnenstrahl, der mich geblendet -
Alles hat am Ende sich gelohnt.

Jedes Feuer, das mein Herz gefangen, 
Jede Sorge, die mein Herz beschlich -
War's oft schwer, so ist's ja doch gegangen.
Narben blieben, doch es lohnte sich.

Unser Leben ist nicht leicht zu tragen.
Nur wer fest sein Herz in Händen hält, 
Hat die Kraft, zum Leben Ja zu sagen
Und zum Kampf für eine neue Welt.

Jeder Tag ist in mein Herz geschlossen,
Der auch mich zu diesem Dienst beschied.
Was ich singe, sing ich den Genossen,
Ihre Träume gehen durch mein Lied.
\end{guitar}


\end{document}
